\section{Related Work}
\label{sec:relwork}
Proliferation of social media platforms such as \emph{Twitter},
\emph{pinterest}, \emph{facebook}, \emph{tumblr} has led to their
application to a wide array of tasks including mental health
assessment~\cite{twitter:anorexia,twitter:violence,twitter:mental}, 
inferring political affiliation~\cite{twitter:elec,twitter:elec2,twitter:elec3,twitter:elec4},
brand perception~\cite{twitter:brand,twitter:brand2} etc.

Social media, especially Twitter, are good sources of personal
health~\cite{twitter:health,twitter:health2,twitter:health3,twitter:health4}.
Previous studies on public health surveillance have attempted to
uncover ailment topics on online
discourse~\cite{DBLP:conf/nips/ChemuduguntaSS06,atam2} or model the
evolution of general topics~\cite{DBLP:conf/kdd/WangAB12}. In this
paper, we combine the best of both worlds which leads to the discovery
of \emph{disease-\changes} for social-media active regions. We model
the evolution of diseases within \changes and obtain significant
improvement over the previous state of the art for public health
surveillance using social media.

We propose TM-ATAM, an approach that builds on the previously proposed
TM-LDA for modeling general topic evolution over
time~\cite{DBLP:conf/kdd/WangAB12}. Just like TM-LDA, TM-ATAM learns
topic transitions over time and not topic trends. Such transitions the
purpose of answering questions such as people talk about fever before
talking about stomach ache. Other complementary approaches that learn
the dynamicity of word distributions or topic trends have been
proposed. That is the case of~\cite{DBLP:conf/icml/BleiL06} that
models topic evolution over time as a discrete chain-style process
where each piece is modeled using
LDA. In~\cite{DBLP:conf/kdd/WangM06}, the authors propose a method
that learns changing word distributions of topics over time and
in~\cite{DBLP:conf/icdm/LinMHJD11}, the authors leverage the structure
of a social network to learn how topics temporally evolve in a
community. Finally, in~\cite{DBLP:conf/wsdm/SahaS12}, Non-negative
Factorization is used for learning topic trends. Exploring the
applicability of those complimentary approaches to the evolution of
health topics in tweets, is a promising research direction.




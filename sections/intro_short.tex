\section{Introduction}
Social media has become a major source of information for analyzing
many aspects of daily life. In particular, public health monitoring can
be conducted on Twitter to measure the well-being of different geographic
populations \cite{atam2}. The ability to model
transitions for ailments and detect statements such as ``people talk
about smoking and cigarettes before talking about respiratory
problems'', or ``people talk about headaches and stomach ache in any
order'', has a range of applications in syndromic surveillance such as
measuring behavioral risk factors and triggering public health
campaigns. 

Popular probabilistic topic modeling methods such as Latent Dirichlet Allocation~\cite{lda} and
pLSA~\cite{plsi} have a long history of successful application to news articles and academic abstracts.
However, the short length of social media posts such as tweets  poses serious
challenges to the efficacy of such methods~\cite{DBLP:conf/ecir/ZhaoJWHLYL11}.
Dedicated methods, such as the Ailment Topic Aspect Model (ATAM), have thus 
been proposed to discover ailments from tweets~\cite{atam2}.

While the primary goal of probabilistic topic modeling is to learn topic models, an equally interesting objective is to examine \emph{topic transitions}.
A temporal extention to LDA (\tmlda) was hence developed for discovering 
the evolution of general-purpose topics in tweets~\cite{DBLP:conf/kdd/WangAB12}. 
In this paper, we examine the feasibility of measuring and predicting ailment
transitions in Twitter, by combining ATAM and TM-LDA into a new model,
coined \tmatam. Our model is different from dynamic
topic models such as~\cite{DBLP:conf/icml/BleiL06, DBLP:conf/kdd/WangM06}, as it is designed to learn topic
transition patterns from temporally-ordered posts, while dynamic topic
models focus on changing word distributions of topics over
time. \tmatam learns transition parameters by minimizing the prediction error on ailment 
distributions of consecutive periods at different temporal and geographic 
granularities.

The effectiveness of \tmatam requires to carefully model two key granularities,
temporal and geographic. A temporal granularity that is too-fine may result in
sparse and spurious transitions whereas a too-coarse one could miss
valuable ailment transitions. Similarly, a too-fine geographic
granularity may produce false positives and a too coarse one may cover
a user population that is exposed to different weather conditions and
miss meaningful transitions. Our experiments on a corpus of more than $500K$ health-related and geo-localized tweets
collected over a period of 8 months, show that \tmatam outperforms \atam,
 \tmlda and \lda in estimating temporal health-related topic transitions of different
geographic populations. The health-related topic transitions we unveiled can be broadly classified in 2 kinds: {\em \selftransitions} are those where a health-related topic
is mentioned continuously. {\em One-way-transitions} cover the case where
some topics are discussed after others.
For example, our study of tweets from Arizona revealed many self-transitions such as
headaches and body pain. On the other hand, tweets about smoking,
drugs and cigarettes in California, are followed by respiratory
ailments.

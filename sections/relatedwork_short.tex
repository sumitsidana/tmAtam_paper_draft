\section{Related Work}
\label{sec:relwork}
Social media has been used for a wide array of tasks including mental health
assessment~\cite{twitter:anorexia,twitter:violence,twitter:mental}, 
inferring political affiliation~\cite{twitter:elec,twitter:elec2,twitter:elec3,twitter:elec4}, and 
brand perception~\cite{twitter:brand,twitter:brand2}.
Previous studies on syndromic surveillance have attempted to uncover ailment topics in online
discourse~\cite{DBLP:conf/nips/ChemuduguntaSS06,atam2} or model the
evolution of general topics~\cite{DBLP:conf/kdd/WangAB12}. In this
paper, we combine the best of both worlds which leads to the discovery
of \emph{\texttt{ailment}}-\seasons for social-media active regions. %% We model
%% the evolution of ailments within \seasons and obtain significant
%% improvement over the previous state of the art for public health
%% surveillance using social media.

Our approach, TM-ATAM, builds on 
TM-LDA for modeling general topic evolution over
time~\cite{DBLP:conf/kdd/WangAB12}. Just like TM-LDA, TM-ATAM learns
topic transitions over time and not topic trends. Other complementary approaches that learn
the dynamicity of word distributions or topic trends have been
proposed such as~\cite{DBLP:conf/icml/BleiL06, DBLP:conf/icdm/LinMHJD11, DBLP:conf/kdd/WangM06}. %% That is also the case of~\cite{DBLP:conf/icml/BleiL06} that
%% models topic evolution over time as a discrete chain-style process
%% where each piece is modeled using LDA. 

In~\cite{DBLP:conf/icdm/DermoucheVKL14}, the authors model topics and their sentiments over time. Topic, sentiment and time are considered random variables. However, since different timestamps may be generated for every word in a single document, a multi-nomial distribution for modeling time is adopted. This choice may not be the best to model time. In \cite{DBLP:journals/ijcv/VaradarajanEO13}, documents are modeled as recurrent sequential patterns called motifs using a generative process. The contribution of the model is 3-fold: capture temporal order of words, detect recurrent and concurrent temporal activities and use a sparsity constraint. However, no prior distribution is used which gives LDA advantage over PLSA.
In~\cite{DBLP:conf/kdd/WangM06}, the authors propose a method
that learns changing word distributions of topics over time and
in~\cite{DBLP:conf/icdm/LinMHJD11}, the authors leverage the structure
of a social network to learn how topics temporally evolve in a
community. Finally, in~\cite{DBLP:conf/wsdm/SahaS12}, non-negative
factorization is used for learning topic trends. 

Exploring the
applicability of those complimentary approaches to the evolution of
health topics in tweets, is a promising research direction.
